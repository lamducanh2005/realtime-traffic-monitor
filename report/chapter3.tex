\chapter{Phương pháp đề xuất}

\section{Kiến trúc hệ thống}

Hệ thống giám sát giao thông dựa trên công nghệ dữ liệu lớn và học sâu bao gồm 3 thành phần chính:
Thiết bị biên (Edge Devices Layer), Trung tâm dữ liệu và xử lý (Processing and Data Central Layer) và Trình giám sát (Monitoring Layer).

\subsection{Thiết bị biên}

Thiết bị biên là các thiết bị thu nhận hoặc phát sinh dữ liệu trực tiếp tại hiện trường.
Đối với hệ thống giám sát giao thông, thiết bị biên có thể bao gồm các camera giám sát, hệ thống đèn tín hiệu, cảm biến giao thông.
Với quy mô và phạm vi của dự án này, hệ thống các thiết bị biên tương đương với hệ thống các camera giám sát được đặt trên cao,
tại các tuyến đường hoặc nút giao của thành phố.

Tùy vào nhu cầu thực hiện các nhiệm vụ mà các thiết bị biên sẽ có phần cứng phù hợp và các chương trình tính toán được lập trình sẵn.
Dự án này thực hiện với giả sử các camera sẽ có độ phân giải HD, được đặt một góc quay chính diện và phù hợp tại các con đường hoặc nút giao,
có phần cứng phù hợp để tích hợp các chương trình xử lý cơ bản.

Trong hệ thống này, các camera sẽ tiến hành ghi lại các khung hình, xử lý ảnh cơ bản, áp dụng mô hình theo dõi vật thể cơ bản
và tiến hành truyền dữ liệu thời gian thực theo từng khung hình về trung tâm xử lý. Để xử lý dữ liệu streaming theo thời gian thực,
các camera sẽ không gửi trực tiếp đến trung tâm xử lý mà thực hiện gửi dữ liệu đến các topic trong Kafka, giao cho Kafka thực hiện phân phối dữ liệu.




\subsection{Trình giám sát}

Trình giám sát là ứng dụng cuối được sử dụng bởi các giám sát viên. Trong hệ thống giám sát giao thông thời gian thực, trình giám sát là đầu ra của hệ thống, hiển thị được hình ảnh truyền phát từ các camera,
thông báo các sự kiện diễn ra và phân tích tình trạng giao thông, tất cả trong thời gian thực. Trình giám sát sẽ nhận dữ liệu streaming theo thời gian thực mà trung tâm xử lý gửi đến Kafka,
hiển thị trực quan trên màn hình giao diện người dùng.



\subsection{Truyền phát}

Tầng truyền phát là cấu nối trung gian giữa các thiết bị biên, trung tâm xử lý và trình giám sát.
Với hệ thống yêu cầu xử lý dữ liệu lớn thì công việc truyền phát càng quan trọng và có ý nghĩa.
Trong hệ thống giám sát giao thông, dữ liệu được sinh ra liên tục từng khung hình, từng sự kiện (biển số, tốc độ, vi phạm, mật độ,...),
nên hệ thống cần một cơ chế truyền phát liên tục, bền vững, an toàn, chịu tải cao và tránh phụ thuộc giữa các thành phần.
Công việc truyền phát trong hệ thống của chúng tôi sử dụng nền tảng Apache Kafka.

Kafka giao tiếp với cả mặt trước (dữ liệu thô từ camera), phần giữa (dữ liệu xử lý trung gian) và mặt sau (dữ liệu đã xử lý gủi đến trình giám sát và cơ sở dữ liệu) của hệ thống.
Việc truyền phát dữ liệu của Kafka trong hệ thống được minh họa tại Hình \ref{fig:kafka-role}.

\begin{figure}[H]
    \centering
    \includegraphics[width=1\textwidth]{images/Kafka role.pdf}
    \caption[Sơ đồ truyền phát dữ liệu của Kafka trong hệ thống]{Sơ đồ truyền phát dữ liệu của Kafka với các thành phần khác trong hệ thống. Kafka đóng vai trò là nơi nhận phát dữ liệu trong hệ thống giám sát giao thông thời gian thực. }
    \label{fig:kafka-role}
\end{figure}

\newpage
Ở phía thiết bị biên, mỗi camera chỉ thực hiện gửi các khung hình và thông tin đã được xử lý sơ bộ đến Kafka mà không cần kết nối trực tiếp với trung tâm xử lý.
Điều này tách biệt giữa nguồn dữ liệu và quá trình xử lý, mỗi camera có thể hoạt động độc lập và không bị ảnh hưởng bởi trạng thái của các thành phần khác, tránh buộc camera phải kết nối với nhiều dịch vụ khác nhau.
Kafka chịu trách nhiệm thu thập luồng dữ liệu lớn từ nhiều camera cùng lúc, có cơ chế đệm để tránh mất dữ liệu khi hàng đợi bị ùn tắc.


Ở trung tâm xử lý, Apache Flink tiêu thụ dữ liệu streaming từ Kafka để thực hiện các tác vụ phân tích theo thời gian thực.
Kafka giúp Flink có thể mở rộng theo chiều ngang và duy trì khả năng xử lý ổn định ngay cả khi dữ liệu tăng đột biến.
Kafka cũng đảm bảo dữ liệu được phân phối theo đúng thứ tự và có thể phục hồi khi xảy ra sự cố.


Với các dữ liệu đầu ra sau khi xử lý, bao gồm biển số, sự kiện giao thông, kết quả phân tích giao thông, Kafka đảm nhiệm tự động phân phối chúng đến trình giám sát hay cơ sở dữ liệu.
Trình giám sát không cần truy vấn liên tục đến trung tâm xử lý hay cơ sở dữ liệu, vì Kafka đã đẩy luồng dữ liệu trực tiếp theo thời gian thực.
Điều này đảm bảo trình giám sát có thể hiển thị thời gian thực, độ trễ thấp, hoạt động ổn định và không ảnh hưởng lẫn nhau.


\subsection{Xử lý luồng dữ liệu}

\begin{figure}[H]
    \centering
    \includegraphics[width=0.95\textwidth]{images/Flink role.pdf}
    \caption[Sơ đồ luồng dữ liệu trong hệ thống]{Sơ đồ luồng dữ liệu đi trong hệ thống. Apache Flink ở trung tâm có nhiệm vụ xử lý các luồng dữ liệu lớn được truyền đi liên tục trong thời gian thực.}
    \label{fig:flink-role}
\end{figure}




\subsection{Cơ sở dữ liệu}

\section{Phân phối, xử lý và lưu trữ dữ liệu}

\subsection{Kafka}

(1) Cài đặt Kafka với một cụm gồm 3 brokers, được quản lý bởi ZooKeeper. Sử dụng KafkaUI để theo dõi trực quan. Tất cả đều sử dụng docker.

(2) Nêu ra các topic, vai trò nghiệp vụ, các thiết đặt.

\subsection{Luồng xử lý dữ liệu chính}

(1) Các luồng xử lý dữ liệu cần phải vẽ được biểu đồ flow

\subsection{Lưu trữ trong cơ sở dữ liệu}

(1) Lưu trữ dữ liệu trong MongoDB. Sử dụng docker.

(2) Nêu ra các schema, định dạng các dữ liệu được lưu trữ.


\section{Mô hình học sâu}


